% Options for packages loaded elsewhere
\PassOptionsToPackage{unicode}{hyperref}
\PassOptionsToPackage{hyphens}{url}
%
\documentclass[
]{book}
\usepackage{amsmath,amssymb}
\usepackage{lmodern}
\usepackage{ifxetex,ifluatex}
\ifnum 0\ifxetex 1\fi\ifluatex 1\fi=0 % if pdftex
  \usepackage[T1]{fontenc}
  \usepackage[utf8]{inputenc}
  \usepackage{textcomp} % provide euro and other symbols
\else % if luatex or xetex
  \usepackage{unicode-math}
  \defaultfontfeatures{Scale=MatchLowercase}
  \defaultfontfeatures[\rmfamily]{Ligatures=TeX,Scale=1}
\fi
% Use upquote if available, for straight quotes in verbatim environments
\IfFileExists{upquote.sty}{\usepackage{upquote}}{}
\IfFileExists{microtype.sty}{% use microtype if available
  \usepackage[]{microtype}
  \UseMicrotypeSet[protrusion]{basicmath} % disable protrusion for tt fonts
}{}
\makeatletter
\@ifundefined{KOMAClassName}{% if non-KOMA class
  \IfFileExists{parskip.sty}{%
    \usepackage{parskip}
  }{% else
    \setlength{\parindent}{0pt}
    \setlength{\parskip}{6pt plus 2pt minus 1pt}}
}{% if KOMA class
  \KOMAoptions{parskip=half}}
\makeatother
\usepackage{xcolor}
\IfFileExists{xurl.sty}{\usepackage{xurl}}{} % add URL line breaks if available
\IfFileExists{bookmark.sty}{\usepackage{bookmark}}{\usepackage{hyperref}}
\hypersetup{
  pdftitle={TensorFlow Lite for Microcontroller (TFLM)},
  pdfauthor={Marconi Jiang},
  hidelinks,
  pdfcreator={LaTeX via pandoc}}
\urlstyle{same} % disable monospaced font for URLs
\usepackage{longtable,booktabs,array}
\usepackage{calc} % for calculating minipage widths
% Correct order of tables after \paragraph or \subparagraph
\usepackage{etoolbox}
\makeatletter
\patchcmd\longtable{\par}{\if@noskipsec\mbox{}\fi\par}{}{}
\makeatother
% Allow footnotes in longtable head/foot
\IfFileExists{footnotehyper.sty}{\usepackage{footnotehyper}}{\usepackage{footnote}}
\makesavenoteenv{longtable}
\usepackage{graphicx}
\makeatletter
\def\maxwidth{\ifdim\Gin@nat@width>\linewidth\linewidth\else\Gin@nat@width\fi}
\def\maxheight{\ifdim\Gin@nat@height>\textheight\textheight\else\Gin@nat@height\fi}
\makeatother
% Scale images if necessary, so that they will not overflow the page
% margins by default, and it is still possible to overwrite the defaults
% using explicit options in \includegraphics[width, height, ...]{}
\setkeys{Gin}{width=\maxwidth,height=\maxheight,keepaspectratio}
% Set default figure placement to htbp
\makeatletter
\def\fps@figure{htbp}
\makeatother
\setlength{\emergencystretch}{3em} % prevent overfull lines
\providecommand{\tightlist}{%
  \setlength{\itemsep}{0pt}\setlength{\parskip}{0pt}}
\setcounter{secnumdepth}{5}
\usepackage{booktabs}
\ifluatex
  \usepackage{selnolig}  % disable illegal ligatures
\fi
\usepackage[]{natbib}
\bibliographystyle{apalike}

\title{TensorFlow Lite for Microcontroller (TFLM)}
\author{Marconi Jiang}
\date{2021-03-21}

\begin{document}
\maketitle

{
\setcounter{tocdepth}{1}
\tableofcontents
}
\hypertarget{ux524dux8a00}{%
\chapter{前言}\label{ux524dux8a00}}

自從 2020 年參加了台灣人工智慧學校的第一屆 Edge AI 技術專班後, 對於 edge AI 比較有點概念, 重讀 \href{https://www.books.com.tw/products/0010865580?fbclid=IwAR1xgb-0LStRrHXajKT-u3uP982d0p8PQaDpNfhxHAzuCbATi6VOjYcwo2k}{TinyML:TensorFlow Lite機器學習} 一書, 於 2021 春節期間比較有空時, 找出手上的 STM32 NUCLEO\_F103RB 板子, 確認程式的可行性, 分享在 \href{https://www.facebook.com/groups/edgeaitw/permalink/2841368276150520/}{FB 的 Edge AI Taiwan邊緣智能交流區}, 後續有臉友聯繫, 希望可以跟學生分享, 就開始了 TensorFlow Lite for Microntroller 更深入的探討, 同時, 也開啟了對 TensorFlow open source 的小小貢獻.

\hypertarget{chap-intro}{%
\chapter{Introducing TensorFlow Lite for Microcontroller (TFLM)}\label{chap-intro}}

Google 推出 \textbf{TensorFlow} 框架, 可以在 CPU/GPU/TPU 上執行 training 的任務, 考慮手機上的侷限的運算能力不如 PC, 接著推出 \href{https://www.tensorflow.org/lite/tutorials}{Tensorflow Lite} 的機器學習框架於手機, 至於 IoT 裝置上用的是 ARM M 系列及其它 embedded MCU (microcontrollers) 的算力以及侷限的 ROM/RAM 空間, 也針對這類型的開發板陸續推出支援機器學習 inference 的 \href{https://www.tensorflow.org/lite/microcontrollers}{TensorFlow Lite for Microcontrollers} (TFLM) 框架, 也出書推廣
\href{https://www.books.com.tw/products/0010865580?fbclid=IwAR1xgb-0LStRrHXajKT-u3uP982d0p8PQaDpNfhxHAzuCbATi6VOjYcwo2k}{TinyML:TensorFlow Lite 機器學習}.

截至目前 (2021 3月) 為止, TFLM 支援的硬體已經有下列的開發板 (書中只有介紹前三種):

\begin{itemize}
\tightlist
\item
  Arduino Nano 33 BLE Sense\\
\item
  SparkFun Edge\\
\item
  STM32F746 Discovery kit\\
\item
  Adafruit EdgeBadge\\
\item
  Adafruit TensorFlow Lite for Microcontrollers Kit\\
\item
  Adafruit Circuit Playground Bluefruit\\
\item
  Espressif ESP32-DevKitC\\
\item
  Espressif ESP-EYE\\
\item
  Wio Terminal: ATSAMD51\\
\item
  Himax WE-I Plus EVB Endpoint AI Development Board\\
\item
  Synopsys DesignWare ARC EM Software Development Platform
\end{itemize}

而介紹的範例則有

\begin{itemize}
\tightlist
\item
  \textbf{Hello World} - 示範 TensorFlow Lite for Microcontrollers 的基本功能, 用機器學習的 inference 來推算出 sine 的值

  \begin{itemize}
  \tightlist
  \item
    Tutorial using any supported device\\
  \end{itemize}
\item
  \textbf{Micro speech} - 從麥克風接收語音, 偵測 wake words: ``yes'', ``no'', ``silence'' 及 ``unkown''

  \begin{itemize}
  \tightlist
  \item
    Tutorial using SparkFun Edge\\
  \end{itemize}
\item
  \textbf{Magic wand} - Captures accelerometer data to classify three different physical gestures

  \begin{itemize}
  \tightlist
  \item
    Tutorial using Arduino Nano 33 BLE Sense\\
  \end{itemize}
\item
  \textbf{Person detection} - Captures camera data with an image sensor to detect the presence or absence of a person
\end{itemize}

\hypertarget{chap-hello-world}{%
\chapter{Hello world 示範程式}\label{chap-hello-world}}

Here is a review of existing methods.

\hypertarget{methods}{%
\chapter{Methods}\label{methods}}

We describe our methods in this chapter.

\hypertarget{applications}{%
\chapter{Applications}\label{applications}}

Some \emph{significant} applications are demonstrated in this chapter.

\hypertarget{example-one}{%
\section{Example one}\label{example-one}}

\hypertarget{example-two}{%
\section{Example two}\label{example-two}}

\hypertarget{final-words}{%
\chapter{Final Words}\label{final-words}}

We have finished a nice book.

  \bibliography{book.bib,packages.bib}

\end{document}
